\section{Generarea si verificarea unui circuit}

\subsection{Alegerea unui circuit arbitrar}
Se va folosi circuitul din figura \ref{fig:circgay}.
Un potential arbore normal este format din latura cu rezistenta $R_1$ si latura cu SIT $E_5$.

\begin{figure}
\begin{tikzpicture}
    % x node set with absolute coordinates
    \node (0) at (0,4) {$V_0$};
    \node (1) at (4,4) {$V_1$};
    \node (2) at (2,0) {$V_2$};

    % y node set relative to x.
    % Locations can be:
    % right,left,above,below,
    % above left,below right, etc

    % Directed edge
    \path[red, line width = 2] (0) edge node[above, el] {$U_1$} (1);
    \path (1) edge node[above, el] {$U_2$} (2);
    \path (2) edge[bend left = 60] node[above, el] {$U_3$} (0);
    \path (1) edge[bend left = 60] node[above, el] {$U_4$} (2);
    \path[red, line width = 2] (0) edge node[above, el] {$U_5$} (2);

    % Bidirected edge
\end{tikzpicture}
\caption{Graficul Tensiunilor}
\end{figure}

\begin{figure}
\begin{tikzpicture}
    % x node set with absolute coordinates
    \node (0) at (0,4){$(0)$};
    \node (1) at (4,4) {$(1)$};
    \node (2) at (2,0) {$(2)$};

    % y node set relative to x.
    % Locations can be:
    % right,left,above,below,
    % above left,below right, etc

    % Directed edge
    \path[red, line width = 2] (1) edge node[above, el] {$I_1$} (0);
    \path (1) edge node[above, el] {$I_2$}(2);
    \path (0) edge[bend right = 60] node[above, el] {$I_3$}(2);
    \path (2) edge[bend right = 60] node[above, el] {$I_4$}(1);
    \path[red, line width = 2] (2) edge node[above, el] {$I_5$}(0);

    % Bidirected edge
\end{tikzpicture}
\caption{Graficul Curentilor}
\end{figure}

\begin{equation}
\left\{
\begin{array}{ccl} % c - centrat, l - aliniat la stanga
I_1 & = & 2 \\
I_2 & = & 1 \\
I_3 & = & 1 \\
I_4 & = & 3 \\
I_5 & = & -1
\end{array}  
\right. \nonumber % marcarea inchiderii - falsa; blocarea numerotarii
\end{equation}

\begin{equation}
\left\{
\begin{array}{ccl} % c - centrat, l - aliniat la stanga
U_1 & = & -1 \\
U_2 & = & 2 \\
U_3 & = & -1 \\
U_4 & = & 2 \\
U_5 & = & 1
\end{array}  
\right. \nonumber % marcarea inchiderii - falsa; blocarea numerotarii
\end{equation}

\begin{figure}
\begin{center}
\begin{circuitikz}[scale=1.4,european resistors,american inductors]
\draw (2,0) -- (0,0) to [R, l = $R_3$, -*] (0,4) to[R, l = $R_1$, -*] (4,4);
\draw (4,0) to[romanianCurrentSource, l = $J_4$] (4, 4);
\draw (4,0) -- (2,0) to[romanianVoltageSource, l = $E_5$, *-] (0,4);
\draw (2,0) to [R, l = $R_2$] (4,4);
\end{circuitikz}
\caption{$R_1 = 0.5\Omega, R_2 = 2\Omega, R_3 = 1\Omega, E_5 = 1V, J_4 = 3A$}
\label{fig:circgay}
\end{center}
\end{figure}

\subsection{Teorema lui Tellegen}
$P = U_2I_2 - U_1I_1 - U_3I_3 - U_4I_4 - U_5I_5$ \\ 
$P = 2 * 1 - (-1) * 2 - (-1) * 1 - 2 * 3 - 1 * (-1) = 0$

\subsection{Bilantul Puterilor}
$P_G = E_5I_5 + U_4J_4 = 1 * (-1) + 3 * 2 = 5$\\
$P_R = R_1I_1^2 + R_2I_2^2 + R_3I_3^2 = 0.5 * 4 + 2 * 1 + 1 * 1 = 5$



