\newpage
\section{Metode sistematice eficiente}

$N = 3$, $L = 5$, $n_{SIC} = 1$, $n_{SIT} = 1$
\begin{table}[ht]  %preferinte: mai intai h - here apoi t- top, htb-, b-bottom
\centering
\caption{Analiza complexitatii}
\label{tab:exemplu_tabel}       % eticheta trebuie sa fie unica

\begin{tabular}{|p{5cm}||c|l|r|p{4cm}|}  \hline  % l = left, c = center, r = right
Metoda & numar de ecuatii  \\ \hline \hline
Kirchhoff clasic & $2L = 10$\\\hline
Kirchhoff in curenti & $L - N + 1 = 3$\\\hline
Kirchhoff in tensiuni  & $N - 1 = 2$\\  \hline
Curenti de coarde & $L - N + 1 - n_{SIC} = 2$\\\hline
Tensiuni in ramuri & $N - 1 - n_{SIT} = 1$\\\hline
\end{tabular}
\end{table}

Vom rezolva circuitul cu metoda potentialelor nodurilor.

\begin{equation}
\left\{
\begin{array}{ccl} % c - centrat, l - aliniat la stanga
V_2 & = & 0 \\
V_0 & = & E_5 \\
V_1(\frac{1}{R_1} + \frac{1}{R_2}) - V_0\frac{1}{R_2} - V_2\frac{1}{R_1} & = & I_4
\end{array}  
\right. \nonumber % marcarea inchiderii - falsa; blocarea numerotarii
\end{equation}

Si rezulta

\begin{equation}
\left\{
\begin{array}{ccl} % c - centrat, l - aliniat la stanga
V_0 & = & 1 \\
V_1 & = & 2 \\
V_2 & = & 0
\end{array}  
\right. \nonumber % marcarea inchiderii - falsa; blocarea numerotarii
\end{equation}

deci

\begin{equation}
\left\{
\begin{array}{ccl} % c - centrat, l - aliniat la stanga
I_1 & = & \frac{V_1 - V_0}{R_1} = 2A \\
I_2 & = & \frac{V_1 - V_2}{R_2} = 1A \\
I_3 & = & \frac{V_0 - V_2}{R_3} = 1A \\
I_4 & = & J_4 = 3A \\
I_5 & = & I_3 - I_1 = -1A
\end{array}  
\right. \nonumber % marcarea inchiderii - falsa; blocarea numerotarii
\end{equation}

si 

\begin{equation}
\left\{
\begin{array}{ccl} % c - centrat, l - aliniat la stanga
U_1 & = & V_0 - V_1 = 1V \\
U_2 & = & V_1 - V_2 = 2V \\
U_3 & = & V_2 - V_0 = -1V \\
U_4 & = & V_1 - V_2 = 2V \\
U_5 & = & E_5 = 1V
\end{array}  
\right. \nonumber % marcarea inchiderii - falsa; blocarea numerotarii
\end{equation}
